% Toggles between whether printed as a handout or "bullet-per-slide"
\documentclass{beamer}
% \documentclass[handout]{beamer}
\usepackage[utf8]{inputenc}
\usetheme{Warsaw}
\usecolortheme{uiowa}
\useoutertheme[subsection=false]{smoothbars}

\usepackage{array}
\usepackage{booktabs}
\usepackage{graphics}
\usepackage{natbib}
\usepackage{xcolor}
\usepackage{amsmath}
\usepackage{amssymb}
\usepackage[english]{babel}
\usepackage{caption}
\usepackage{amsfonts}
\usepackage{tikz}
\usepackage{bm}
\usepackage{hyperref}
\input{custom}

% link colors
\hypersetup{
  colorlinks=true,
  citecolor=black,
  linkcolor=.,
  urlcolor=uiblue
}

% ------------------------------------------------------------------
% Title info
% ------------------------------------------------------------------
\title[Collaboration with GitHub \hspace{3mm} Version Control in Practice \hspace{7em} \insertframenumber/\inserttotalframenumber]
{Version Control, GitHub Collaboration,\\ and Reproducible Team Workflow}

\author[Group Project]{Dylan Day, Shuo Li, Trevor Debutch, Shouhardyo Sarkar}
\date{\today}
\institute[The University of Iowa]{The University of Iowa, Department of Statistics and Actuarial Science}

\logo{\includegraphics[width=0.12\textwidth]{iowa-logo.png}}

% ------------------------------------------------------------------
\begin{document}
% ------------------------------------------------------------------

\frame{\titlepage}

% ==============================
\section{Motivation \& Goals}
% ==============================

\begin{frame}{Why Are We Talking About Version Control?}
  \begin{itemize}
    \item We work in groups and need to edit the same files without overwriting each other.
    \item We want a record of \textbf{who changed what and when}.
    \item We want to be able to \textbf{go back in time} if something breaks.
    \item We want our work to be \textbf{reproducible} on any machine.
  \end{itemize}

  \vspace{2mm}
  \textbf{Version control} is the system that tracks changes to files over time so you can recall or restore specific versions later.
\end{frame}

\begin{frame}{What We'll Show in This Talk}
  \begin{itemize}
    \item What Git and GitHub actually do (not just buzzwords).
    \item How teams collaborate using branches, commits, and pull requests.
    \item How to connect a local folder to a GitHub repo and push/pull.
    \item How this supports small coding projects (for class or research).
  \end{itemize}
\end{frame}

% ==============================
\section{Git, GitHub, and Key Concepts}
% ==============================

\begin{frame}{Git vs GitHub}
  \begin{block}{Git}
    A distributed version control system: every person has a full copy of the repository, history and all.
    Fast, offline-friendly.
  \end{block}

  \begin{block}{GitHub}
    A cloud platform \textbf{built on top of Git} that adds:
    \begin{itemize}
      \item remote hosting / backup
      \item pull requests \& code review
      \item issues \& project boards
      \item CI/CD automation
    \end{itemize}
    GitHub turns source code into a collaborative workspace.
  \end{block}
\end{frame}

\begin{frame}{GitHub Glossary}
  \begin{itemize}
    \item \textbf{Repository (repo)}: the project folder on GitHub.
    \item \textbf{Commit}: a saved change with a message.
    \item \textbf{Branch}: a parallel version of the project used to develop features safely.
    \item \textbf{Pull Request (PR)}: a request to merge changes from one branch into another after review.
  \end{itemize}

  \vspace{2mm}
  These concepts let multiple people experiment, review, and merge without breaking \texttt{main}.
\end{frame}

\begin{frame}{Why Use GitHub in a Class Project?}
  \begin{itemize}
    \item \textbf{Collaboration}: everyone can contribute code, documentation, or data without emailing files.
    \item \textbf{History}: you can see exactly how the project evolved.
    \item \textbf{Accountability}: each commit is tied to an author.
    \item \textbf{Safety}: if someone breaks something, you can roll back.
    \item \textbf{Publishability}: you can keep it private or make it public to show future advisors / employers.
  \end{itemize}
\end{frame}

% ==============================
\section{Basic Git Workflow}
% ==============================

\begin{frame}{Core Commands You Will Actually Use}
  \begin{itemize}
    \item \texttt{git clone <repo>} \\
      download a copy of a GitHub repo to your computer
    \item \texttt{git add .} \\
      stage your edits (tell Git which changes you want to commit)
    \item \texttt{git commit -m "message"} \\
      record a snapshot of those changes
    \item \texttt{git pull} and \texttt{git push} \\
      sync with GitHub (pull down others' work, push up yours)
    \item \texttt{git branch}, \texttt{git merge} \\
      work on new ideas in isolation, then merge them back
  \end{itemize}

  \vspace{2mm}
  These actions are the backbone of team development.
\end{frame}

\begin{frame}{Branches and Pull Requests}
  \begin{itemize}
    \item Each teammate can create a new branch (for example, \texttt{feature-ui}).
    \item You do your work there without touching the \texttt{main} branch.
    \item When ready, you open a \textbf{Pull Request} on GitHub:
      \begin{itemize}
        \item show what changed
        \item explain why
        \item request review
      \end{itemize}
    \item After review, the branch is merged into \texttt{main}.
  \end{itemize}

  \vspace{2mm}
  This workflow gives you feedback, quality control, and a clean history instead of chaos.
\end{frame}

% ==============================
\section{Connecting Local and Remote}
% ==============================

\begin{frame}{Step 1: Clone the Repository}
  \begin{itemize}
    \item One person creates a repo on GitHub.
    \item Everyone else runs:
  \end{itemize}

  \vspace{1mm}
  \texttt{git clone https://github.com/owner/RepoName.git}

  \vspace{2mm}
  Now you have a full local copy of the project folder, including history and branches.
\end{frame}

\begin{frame}[fragile]{Step 2: Check the Remote Link}
  \begin{itemize}
    \item Inside the project folder, verify GitHub is set as ``origin'':
  \end{itemize}

  \vspace{1mm}
  \texttt{git remote -v}

  \vspace{2mm}
  Expected:
  \begin{verbatim}
origin  https://github.com/owner/RepoName.git (fetch)
origin  https://github.com/owner/RepoName.git (push)
  \end{verbatim}

  \vspace{2mm}
  If you see that, your local repo is correctly connected to the GitHub repo.
\end{frame}

\begin{frame}{Step 3: Pull and Push}
  \begin{itemize}
    \item \textbf{Pull} before you start editing:
  \end{itemize}

  \vspace{1mm}
  \texttt{git pull origin main}

  \vspace{3mm}
  \begin{itemize}
    \item After editing files:
  \end{itemize}

  \vspace{1mm}
  \texttt{git add .}\\
  \texttt{git commit -m "explain what you changed"}\\
  \texttt{git push origin main}

  \vspace{2mm}
  If push is rejected, it means someone else updated \texttt{main}; run \texttt{git pull} to merge first.
\end{frame}

\begin{frame}{Step 4: Authentication / Permissions}
  \begin{itemize}
    \item GitHub requires you to prove who you are when pushing.
    \item If you see ``Permission denied'' or ``Authentication failed'':
      \begin{itemize}
        \item You may need a Personal Access Token (PAT) instead of a password.
        \item Or you may not have been added as a collaborator yet.
      \end{itemize}
    \item Good news: pulling (reading) is often easier than pushing (writing).
  \end{itemize}
\end{frame}

% ==============================
\section{Team Project Angle}
% ==============================

\begin{frame}{How This Supports a Team Project}
  \begin{itemize}
    \item Everyone can contribute in parallel:
      \begin{itemize}
        \item Code / analysis scripts
        \item README documentation
        \item Example data or test cases
      \end{itemize}
    \item Changes are tracked and credited to specific authors.
    \item The repo becomes a ``single source of truth'' for the project.
    \item You can show the final repo (or a link) as part of your assignment deliverable.
  \end{itemize}
\end{frame}

\begin{frame}{What We Practiced as a Group}
  \begin{itemize}
    \item Setting up a shared GitHub repository.
    \item Cloning to each teammate's laptop.
    \item Making edits on branches and creating pull requests for review.
    \item Resolving merge conflicts.
    \item Documenting everything in \texttt{README.md} so it is reproducible later.
  \end{itemize}
\end{frame}

% ==============================
\section{Takeaways}
% ==============================

\begin{frame}{Key Takeaways}
  \begin{itemize}
    \item Version control = a time machine for your project.
    \item GitHub = collaboration, accountability, and backup.
    \item Branches + pull requests = safe teamwork without stepping on each other's work.
    \item Pull before you push.
    \item Clear commit messages and good README files make your work reusable.
  \end{itemize}
\end{frame}

\begin{frame}{Questions?}
  \centering
  \Large Thank you!
\end{frame}

% ------------------------------------------------------------------
\bibliographystyle{plainnat}
\bibliography{refs}

\end{document}
